\documentclass[12pt]{article}

\usepackage{amsmath, amsfonts, amssymb, natbib, graphicx, rotating, lscape, longtable, booktabs, threeparttable, multirow, afterpage, setspace, tikz, caption, subcaption, bigstrut, mathrsfs, dsfont, titlesec, caption, soul, bbm}
\usepackage[multiple, bottom]{footmisc}
\usepackage[top=1in, bottom=1in, left=1in, right=1in]{geometry}
\usepackage[unicode=true,pdfusetitle,
 bookmarks=true,bookmarksnumbered=false,bookmarksopen=false,
 breaklinks=true,pdfborder={0 0 1},backref=false,colorlinks=true]
 {hyperref}
\usetikzlibrary{patterns, arrows}
\usepackage[draft]{todonotes}

\newcommand{\indep}{\perp \!\!\! \perp}

\hypersetup{citecolor=blue,linkcolor=blue,filecolor=blue,urlcolor=blue}
\newcommand*\multiplefootnoteseparator{%
  \textsuperscript{\multfootsep}\nobreak
}

\let\mfs\multiplefootnoteseparator
\allowdisplaybreaks
\onehalfspacing
\newtheorem{prediction}{Prediction}

\newcommand{\pd}[2]{\frac{\partial #1}{\partial #2}}
\newcommand\independent{\protect\mathpalette{\protect\independenT}{\perp}}
\def\independenT#1#2{\mathrel{\rlap{$#1#2$}\mkern2mu{#1#2}}}
\newcommand{\hiyel}[1]{\colorbox{yellow}{#1}}
\newcommand{\hired}[1]{\colorbox{red}{#1}}

\titlespacing*{\section}{0pt}{5pt}{5pt}
\titlespacing*{\subsection}{0pt}{5pt}{5pt}
\titlespacing*{\subsubsection}{0pt}{5pt}{5pt}
\titlespacing*{\paragraph}{0pt}{5pt}{10pt}
\setlength{\bibsep}{0pt}
\setlength{\textfloatsep}{10pt}
\setlength{\abovedisplayskip}{0pt}
\setlength{\belowdisplayskip}{0pt}
\setlength{\footnotesep}{0pt}

\renewcommand{\topfraction}{.9}	
\renewcommand{\textfraction}{.01}	

\setcounter{MaxMatrixCols}{10}

\newtheorem{theorem}{Theorem}
\newtheorem{acknowledgment}[theorem]{Acknowledgment}
\newtheorem{algorithm}{Algorithm}
\newtheorem{assumption}[theorem]{Assumption}
\newtheorem{axiom}[theorem]{Axiom}
\newtheorem{case}[theorem]{Case}
\newtheorem{claim}[theorem]{Claim}
\newtheorem{conclusion}[theorem]{Conclusion}
\newtheorem{condition}[theorem]{Condition}
\newtheorem{conjecture}[theorem]{Conjecture}
\newtheorem{corollary}{Corollary}
\newtheorem{criterion}[theorem]{Criterion}
\newtheorem{definition}[theorem]{Definition}
\newtheorem{example}{Example}
\newtheorem{exercise}[theorem]{Exercise}
\newtheorem{lemma}{Lemma}
\newtheorem{notation}[theorem]{Notation}
\newtheorem{problem}[theorem]{Problem}
\newtheorem{proposition}{Proposition}
\newtheorem{remark}[theorem]{Remark}
\newtheorem{solution}[theorem]{Solution}
\newtheorem{summary}[theorem]{Summary}
%\usepackage[bold-style=ISO]{unicode-math}

\newenvironment{proof}[1][Proof]{\noindent\textbf{#1.} }{\ \rule{0.5em}{0.5em}}

\begin{document}


\title{School Food Procurement\thanks{We thank...}}
\author{Martin B.\ Hackmann\thanks{UCLA, Department of Economics, CESifo, and NBER} \and Darion Phan\thanks{UCLA} }
\date{\today}

\maketitle

\begin{abstract} This paper studies the impact of food revenues on food procurement, and student outcomes.
	\end{abstract}

\section{Study Goals}

\begin{itemize}
	\item What is the impact of food revenues on food procurement?
	\item What is the impact of food procurement on student outcomes
\end{itemize}

\section{Institutions}
\begin{enumerate}
	\item School Districts
	\item Schools
	\item Food Service Providers
	\item Students
\end{enumerate} 

\section{Data Sources}

\subsection{California}

\begin{enumerate}
	\item California Department of Education
	\begin{itemize}
		\item test scores
		\item physical fitness
		\item demographics
		\item Food procurement... details
	\end{itemize}
	\item Los Angeles School Districts
	\end{enumerate}

\subsection{Texas}

\section{Data Description}

\section{Empirical Evidence}

\section{Figures}

\begin{figure}[tbh]
	\caption{Figure title}\label{fig1}
		%\includegraphics[width=3in]{Figures/Odense_allsignals3.pdf}
\end{figure}

\section{Tables}

\begin{table}[tbh]
    \caption{Table title}
    \label{table1}
        \begin{center}
        %\scalebox{0.85}{\input{Tables/tablefit_prepost.tex}}
        \end{center}
    \textit{\scriptsize{}Note: This table...}{\scriptsize \par}
    \end{table}

\section{Todos:}

\begin{itemize}
	\item Fix 40\% graphs, add to tex file with short description
	\item Add description of the data
	\begin{itemize}
		\item number schools/Districts
		\item years of data
		\item add summary with main variables: rows are variables and columns have summary statistics (number of observations, mean, std, min, max, missing values)
		\item two tables: one at school and year level and one at school district level
	\end{itemize} 
	\item add spending data
	\item add cost of living data
\end{itemize}


\end{document}
